%%%%%%%%%%%%%%%%%%%%%%%%%%%%%%%%%%%%%%%%%%
%                                        %
% Szablon pracy dyplomowej magisterskiej % 
%                                        %
%%%%%%%%%%%%%%%%%%%%%%%%%%%%%%%%%%%%%%%%%%



\documentclass[a4paper,twoside,12pt]{book}
\usepackage[utf8]{inputenc}
\usepackage[T1]{fontenc}
\usepackage{amsmath,amsfonts,amssymb,amsthm}
\usepackage[british,polish]{babel}
\usepackage{indentfirst}
\usepackage{lmodern}
\usepackage{graphicx}
\usepackage{hyperref}
\usepackage{booktabs}
\usepackage{tikz}
\usepackage{pgfplots}
\usepackage{mathtools}
\usepackage[page]{appendix} % toc,
\renewcommand{\appendixtocname}{Dodatki}
\renewcommand{\appendixpagename}{Dodatki}
\renewcommand{\appendixname}{Dodatek}

\usepackage{setspace}
\onehalfspacing


\frenchspacing

\usepackage{listings}
\lstset{
language={},
basicstyle=\ttfamily,
keywordstyle=\lst@ifdisplaystyle\color{blue}\fi,
commentstyle=\color{gray}
}

%%%%%%%%%

%%%% LIST GENERATOR %%%%%%%%%

%\usepackage{tikz}
%\usepackage{manfnt}   % dangerous sign 
\usepackage{color}
\definecolor{brickred}      {cmyk}{0 , 0.89, 0.94, 0.28}

\makeatletter \newcommand \kslistofremarks{\section*{Uwagi} \@starttoc{rks}}
\newcommand\l@uwagas[2]
{\par\noindent \textbf{#2:} %\parbox{10cm}
{#1}\par} \makeatother


\newcommand{\ksremark}[1]{%
{%\marginpar{\textdbend}
{\color{brickred}{[#1]}}}%
\addcontentsline{rks}{uwagas}{\protect{#1}}%
}

\newcommand{\comma}{\ksremark{przecinek}}
\newcommand{\nocomma}{\ksremark{bez przecinka}}
\newcommand{\styl}{\ksremark{styl}}
\newcommand{\ortografia}{\ksremark{ortografia}}
\newcommand{\fleksja}{\ksremark{fleksja}}
\newcommand{\pauza}{\ksremark{pauza `--', nie dywiz `-'}}
\newcommand{\kolokwializm}{\ksremark{kolokwializm}}

%%%%%%%%%%%%%% END OF GENERATOR %%%%%%%%%%%

%%%%%%%%%%%% ZYWA PAGINA %%%%%%%%%%%%%%%
% brak kapitalizacji zywej paginy
\usepackage{fancyhdr}
\pagestyle{fancy}
\fancyhf{}
\fancyhead[LO]{\nouppercase{\it\rightmark}}
\fancyhead[RE]{\nouppercase{\it\leftmark}}
\fancyhead[LE,RO]{\it\thepage}


\fancypagestyle{tylkoNumeryStron}{%
\fancyhf{}
\fancyhead[LE,RO]{\it\thepage}
}

\fancypagestyle{NumeryStronNazwyRozdzialow}{%
\fancyhf{}
\fancyhead[LO]{\nouppercase{\it\rightmark}}
\fancyhead[RE]{\nouppercase{\it\leftmark}}
\fancyhead[LE,RO]{\it\thepage}
}


%%%%%%%%%%%%% OBCE WTRETY  
\newcommand{\obcy}[1]{\emph{#1}}
\newcommand{\ang}[1]{{\selectlanguage{british}\obcy{#1}}}
%%%%%%%%%%%%%%%%%%%%%%%%%%%%%

% polskie oznaczenia funkcji matematycznych
\renewcommand{\tan}{\operatorname {tg}}
\renewcommand{\log}{\operatorname {lg}}

% jeszcze jakies drobiazgi

\newcounter{stronyPozaNumeracja}

\newcommand{\hcancel}[1]{%
\tikz[baseline=(tocancel.base)]{
\node[inner sep=0pt,outer sep=0pt] (tocancel) {#1};
\draw[red] (tocancel.south west) -- (tocancel.north east);
}%
}%

\newcommand{\miesiac}{%
\ifcase\the\month
\or styczeń% 1
\or luty% 2
\or marzec% 3
\or kwiecień% 4
\or maj% 5
\or czerwiec% 6
\or lipiec% 7
\or sierpień% 8
\or wrzesień% 9
\or październik% 10
\or listopad% 11
\or grudzień% 12
\fi}


%%%%%%%%%%%%%%%%%%%%%%%%%%%%%%%%%%%%%%%%%%%%%%
%%%%%%%%%%%%%%%%%%%%%%%%%%%%%%%%%%%%%%%%%%%%%%
%%%%%%%%%%%%%%%%%%%%%%%%%%%%%%%%%%%%%%%%%%%%%%
%%%%%%%%%%%%%%%%%%%%%%%%%%%%%%%%%%%%%%%%%%%%%%
%%%%%%%%%%%%%%%%%%%%%%%%%%%%%%%%%%%%%%%%%%%%%%
%%%%%%%%%%%%%%%%%%%%%%%%%%%%%%%%%%%%%%%%%%%%%%
%%%%%%%%%%%%%%%%%%%%%%%%%%%%%%%%%%%%%%%%%%%%%%
%%%%%%%%%%%%%%%%%%%%%%%%%%%%%%%%%%%%%%%%%%%%%%


\newcommand{\autor}{Mateusz Trzeciak}
\newcommand{\promotor}{dr hab. inż. Karolina Nurzyńska}
\newcommand{\tytul}{Określenie wieku twarzy na podstawie tekstury}


\begin{document}
    %\kslistofremarks

    %%%%%%%%%%%%%%%%%%  STRONA TYTULOWA %%%%%%%%%%%%%%%%%%%
    \pagestyle{empty}
    \sffamily

    \noindent

    \begin{center}
        \large
        Politechnika Śląska\\
        Wydział Automatyki, Elektroniki i~Informatyki \\
        kierunek: informatyka
    \end{center}

    \vfill\vfill
    \begin{center}
        \large
        \autor
    \end{center}

    \vfill
    \begin{center}
        \LARGE\bfseries \tytul
    \end{center}

    \vfill
    \begin{center}
        \large
        praca dyplomowa magisterska
    \end{center}

    \vfill\vfill\vfill
    \begin{center}
        \large
        \begin{tabular}{ll}
            promotor: & \promotor \\
            % konsultant: & \konsultant \\ % jezeli nie ma, zakomentowac
        \end{tabular}

    \end{center}

    \vfill
    \begin{center}
        \large
        Gliwice,  \miesiac\ \the\year
    \end{center}

    \cleardoublepage


    \rmfamily
    \normalfont

    %%%%%%%%%%%%%%%%%%%%% oswiadczenie o udostępnianiu pracy dyplomowej %%%%%%%%%%%%%%%%%%%
    \cleardoublepage

    \begin{flushright}
        załącznik nr 2 do zarz. nr 97/08/09
    \end{flushright}

    \vfill

    \begin{center}
        \Large\bfseries Oświadczenie
    \end{center}

    \vfill

    Wyrażam zgodę / Nie wyrażam zgody* na udostępnienie mojej pracy dyplomowej / rozprawy doktorskiej*.

    \vfill

    Gliwice, dnia \today

    \vfill

    \rule{0.5\textwidth}{0cm}\dotfill

    \rule{0.5\textwidth}{0cm}
    \begin{minipage}{0.45\textwidth}
    {\begin{center}
         (podpis)
    \end{center}}
    \end{minipage}

    \vfill

    \rule{0.5\textwidth}{0cm}\dotfill

    \rule{0.5\textwidth}{0cm}
    \begin{minipage}{0.45\textwidth}
    {\begin{center}
         \rule{0mm}{5mm}(poświadczenie wiarygodności podpisu przez Dziekanat)
    \end{center}}
    \end{minipage}


    \vfill

    * podkreślić właściwe




    %%%%%%%%%%%%%%%%%%%%% oswiadczenie promotora o spełnieniu wymagań formalnych %%%%%%%%%%%%%%%%%%%
    \cleardoublepage

    \rule{1cm}{0cm}

    \vfill

    \begin{center}
        \Large\bfseries Oświadczenie promotora
    \end{center}

    \vfill

    Oświadczam, że praca „\tytul” spełnia wymagania formalne pracy dyplomowej magisterskiej.

    \vfill



    \vfill

    Gliwice, dnia \today

    \rule{0.5\textwidth}{0cm}\dotfill

    \rule{0.5\textwidth}{0cm}
    \begin{minipage}{0.45\textwidth}
    {\begin{center}
         (podpis promotora)
    \end{center}}
    \end{minipage}

    \vfill

    %\rule{0.5\textwidth}{0cm}\dotfill
    %
    %\rule{0.5\textwidth}{0cm}
    %\begin{minipage}{0.45\textwidth}
    %{\begin{center}\rule{0mm}{5mm}(poświadczenie wiarygodności podpisu przez Dziekanat)\end{center}}
    %\end{minipage}
    %
    %
    %\vfill



    \cleardoublepage


    %%%%%%%%%%%%%%%%%% SPIS TRESCI %%%%%%%%%%%%%%%%%%%%%%
    \pagenumbering{Roman}
    \pagestyle{tylkoNumeryStron}
    \tableofcontents

    %%%%%%%%%%%%%%%%%%%%%%%%%%%%%%%%%%%%%%%%%%%%%%%%%%%%%
    \setcounter{stronyPozaNumeracja}{\value{page}}
    \mainmatter
    \pagestyle{NumeryStronNazwyRozdzialow}

    %%%%%%%%%%%%%% wlasciwa tresc pracy %%%%%%%%%%%%%%%%%

    \chapter{Wstęp}

    %\begin{itemize}
    %\item wprowadzenie w problem/zagadnienie
    %\item osadzenie problemu w dziedzinie
    %\item cel pracy
    %\item zakres pracy
    %\item zwięzła charakterystyka rozdziałów
    %\item jednoznaczne określenie wkładu autora
    %\end{itemize}
    Wiek jest cechą, którą niełatwo człowiekowi odczytać z czyjejś twarzy. Dla komputera rozpoznawanie wieku jest
    trudniejsze niż dla człowieka.
    Dlatego do wyznaczania wieku z pomocą programu komputerowego należy podchodzić z dystansem. Mimo trudności programiści
    i naukowcy udoskonalają algorytmy,
    tak aby ocena wieku danej osoby była coraz dokładniejsza.

    Istnieje wiele sposobów wyznaczania wieku.
    Większość metod skupia się na analizie tekstury twarzy. Idąc dalej - z obrazu danej osoby lub jego części, np tułowia,
    musi zostać wykryta twarz. Wykrycie twarzy na teksturze jest możliwe dzięki algorytmom rozpoznawaniu obrazu.
    Rozpoznawanie obrazu jest stosowane w wizji komputerowej i polega na wyodrębnieniu z obrazu jakichś szczegółów. Mogą
    to być osoby, pojazdy, przedmioty itp. (Rys. \ref{fig.rozpoznawanieObiektow})

    \begin{figure}
        \centering
        \includegraphics[width=11cm]{Obrazy/rozpoznawanieObiektow}
        \caption{Przykład rozpoznawania obiektów na zdjęciu ulicy. \cite{rozpoznawanieObiektow}}
        \label{fig.rozpoznawanieObiektow}
    \end{figure}

    Można znaleźć wiele witryn internetowych, które udostępniają interfejsy programistyczne umożliwiające zaimplementowanie
    rozpoznawania wieku z obrazu.
    Istnieją algorytmy przetwarzania obrazu, które oprócz wieku wyznaczają z pewnym prawdopodobieństwem płeć danej osoby.
    Oprócz płci mogą one także wyznaczyć minę oraz czy dana osoba nosi okulary.

    Z weryfikacją wieku danej osoby można się spotkać przed wejściem do niektórych miejsc, tj. klub nocny. Większość osób
    musi okazać ważny dowód osobisty,
    co generuje duże kolejki do wejścia. Aplikacje analizujące wiek na podstawie obrazu twarzy z kamery przed wejściem
    do takich miejsc znacząco usprawniłyby weryfikację wieku.
    Rozpoznawanie wieku może być wykorzystywane przy analizie średniego wieku ludzi w jakimś miejscu np. podczas demonstracji.

    Wiele gier posiada treści nieodpowiednie dla młodszych użytkowników. Możliwe jest stosowanie technologii wykrywania
    wieku użytkownika przed udostępnieniem mu treści, która wymaga odpowiedniego wieku.

    Można znaleźć o wiele więcej potencjalnych zastosowań przetwarzania obrazu oraz rozpoznawania wieku na podstawie
    tekstury (obrazu) twarzy. Z biegiem lat z pewnością będzie można zauważyć dalszy rozwój tej dziedziny, która
    opiera się w głównej mierze na sztucznej inteligencji \cite{computerVision}.

    \begin{figure}
        \centering
        \includegraphics[width=11cm]{Obrazy/Faza1.jpg}
        \caption{Faza 1 algorytmu}
        \label{fig.faza1Algorytmu}
    \end{figure}

    \begin{figure}
        \centering
        \includegraphics[width=11cm]{Obrazy/Faza2.jpg}
        \caption{Faza 2 algorytmu}
        \label{fig.faza2Algorytmu}
    \end{figure}

    %    \section{Cel i zakres pracy}
    %    Celem pracy magisterskiej jest stworzenie prostego programu do rozpoznawania wieku na podstawie tekstury twarzy.
    %
    %    Zakres pracy obejmuje:
    %    \begin{itemize}
    %        \item Wybór bazowej metody wyznaczania wieku
    %        \item Stworzenie kilku modyfikacji bazowej metody
    %        \item Opis algorytmów każdej z metod wyznaczania wieku
    %        \item Porównanie wszystkich metod i wybór najlepszej
    %    \end{itemize}
    %
    %    \chapter{Przegląd metod wyznaczania wieku}
    %    \section{Metoda a}
    %    \section{Metoda b}
    %    \section{Metoda wrinkle feature}

    \chapter{Metoda bazowa - wrinkle feature}
    Istnieje wiele metod wyznaczania wieku z obrazu twarzy. Każda z nich opiera się na wykrywaniu zmarszczek,
    zmian koloru skóry lub proporcji twarzy. W omawianej pracy zastosowano metodę korzystającą z wykrywania zmarszczek.
    ~\cite{Kriegman}

    Metoda bazowa została opisana w artykule ,,Age Estimation from Face Image using Wrinkle Features''
    ~\cite{wrinkleFeatures}.
    Wykrywanie wieku dzieli się na kilka faz. Na początku należy wykryć twarz. Zastosowany algorytm wykrywania został
    opisany w sekcji \ref{sec:metodaWykrywaniaTwarzy}
    Następnie należy wyznaczyć strefy zmarszczkowe na twarzy. W artykule \cite{wrinkleFeatures} udowodniono,
    że istnieje kilka konkretnych stref, w których następuje znacząca zmiana ilości zmarszczek wraz z wiekiem.
    Powyższe strefy zostały wymienione w sekcji \ref{sec:wyznaczanieStref}. Sekcja
    \ref{sec:wykrywanieZmarszczek}przedstawia technikę wykrywania zmarszczek znajdujących się w strefach. Wykryte zmarszczki
    pozwalają na obliczenie wrinkle feature dla danej twarzy, zgodnie z opisem w sekcji \ref{sec:wyliczanieWrinkleFeature}.
    W tym miejscu kończy się faza wyznaczania wrinkle feature dla danej osoby. Kolejna faza jest potrzebna do
    znalezienia relacji pomiędzy wrinkle feature a wiekiem. Do tego celu należy zastosować algorytm trenujący, który
    został opisany w sekcji \ref{sec:algorytmTrenowania}. Wynikiem algorytmu trenującego jest zbiór danych, który
    należy pogrupować, tak jak to opisano w sekcji \ref{sec:grupowanieDanych}. Ostatnią fazą algorytmu jest wykrywanie wieku
    na podstawie wyników działania FCM - sekcja \ref{sec:wyznaczanieWieku}.

    %    tutaj trzeba dodac obrazek przedstawiajacy algorytm - najlepiej wraz z miniaturkami jak wyglada obrazek w kazdej
    %    fazie...

    \section{Metoda wykrywania twarzy}\label{sec:metodaWykrywaniaTwarzy}
    W literaturze można odnaleźć wiele metod wykrywania twarzy. Niektóre z nich do skutecznej kwalifikacji
    stosują ekstrację cech z obrazu. Ekstrakcja polega na przekształceniu obrazu do zbioru zmiennych, które
    zostaną później użyte w wykrywaniu obiektu lub obiektów na obrazie. Warto wymienić
    najczęściej stosowane metody służące do ekstrakcji cech \cite{computerVision}:
    \begin{itemize}
        \item Filtry Gabora
        \item Filtry Haara
        \item Funkcje jądra
        \item Analiza składowych górnych
    \end{itemize}
    Algorytm Haar Cascade jest najpopularniejszym algorytmem do wykrywania twarzy w bibliotece OpenCV. Biblioteka OpenCV
    została tutaj wymieniona, ponieważ wykorzystano ją przy wykrywaniu wieku z tekstury twarzy. W związku z powyższym
    do wykrywania twarzy zastosowano algorytm Haar Cascade.

    %    https://towardsdatascience.com/face-recognition-with-opencv-haar-cascade-a289b6ff042a
    %    przez kogo został zaproponowany...
    %    byc moze cos o funkcji kaskadowej
    %    o tym ze OpenCv zawiera wytrenowane algorytmy Haar Cascade ktore wykrywaja ryj, oczy, usta itp
    %    wkleic obrazek z rodzajami filtrow
    %    jak dziala ten filtr
    %    przyklad na gebie emmy watson moze...
    %cosik tam jeszcze

    %    opisac jak dokladnie zaimplementowano to u nas
    %    przejscie na skale szarosci podczas wykrywania... czym ona jest?...
    \subsection{Wstęp do algorytmu Haar Cascade}
    Haar Cascade jest algorytem z dziedziny uczenia maszynowego służacy do wykrywania obiektów w wizji komputerowej.
    Został stworzony przez Paula Viola oraz Michaea Jonsa \cite{violaJones}.
    Algorytm opiera się na zbudowaniu kaskadowej funkcji za pomocą trenowania wielu zdjęć. Zdjęcie są dzielone na
    dwie kategorie - pozytywne oraz negatywne. Na pozytywnych zdjęciach istnieje obiekt, który ma zostać wykryty.
    Analogicznie na negatywnych zdjęciach nie ma tego obiektu.

    Ekstrakcja cech w algorytmie Violi i Jonsa jest realizowana przez filtry Haara. Filtry Haara to prostokątne okienka
    nakładane na obraz, które analizują jasność pikseli \ref{fig.haarRectangles}.
    \begin{figure}
        \centering
        \includegraphics[width=7cm]{Obrazy/Haar_filter_rectangles.jpg}
        \caption{Filtr Haara a) krawędziowy b) liniowy c) szachownica \cite{haar}}
        \label{fig.haarRectangles}
    \end{figure}

    Przed zastosowaniem filtru Haara obraz musi zostać przekształcony do skali szarości. Skala szarości jest rodzajem
    odwzorowania koloru. Wyróżnia się 3 podstawowe tryby koloru:
    \begin{itemize}
        \item kolorowy
        \item monochromatyczny (skala szarości)
        \item czarnobiały
    \end{itemize}
    W niniejszej pracy należało przekształcić każdy obraz z trybu kolorowego na monochromatyczny.

    \subsection{Konwersja do skali szarości}

    Każdy piksel w trybie kolorowym ma określoną reprezentację barwy z określonego modelu. Najczęściej spotykana
    reprezentacja to RGB. Przestrzeń kolorów RGB to złożenie 3 kanałów:

    \begin{itemize}
        \item R - czerwonego z angielskiego Red
        \item G - zielonego z angielskiego Green
        \item B - niebieskiego z angielskiego Blue
    \end{itemize}

    Każdy piksel opisany za pomocą przestrzenie barw RGB ma 3 wartości reprezentujący każdy kanał. Wartość 0 danego
    kanału oznacza brak jasności, natomiast 255 maksymalną jasność.

    \subsection{Kontynuuacja opisu algorytmu Haar Cascade}
    Każde okienko posiada białe oraz czarne prostokąty. Wyznaczana jest suma jasności pikseli w białych prostokątach
    oraz czarnych. Następnie dla każdego okna obliczana jest różnica pomiędzy białymi a czarnymi kwadratami.
    Opisywany algorytm ma zastosowaniu w wykrywaniu krawędzi. Na granicy krawędzi istnieje różnica w jasności pikseli
    \ref{fig.haarEmmaWatson}.

    \begin{figure}
        \centering
        \includegraphics[width=10cm]{Obrazy/haarEmmaWatson.jpg}
        \caption{Filtr Haara nałożony na krawędź twarzy \cite{haar}}
        \label{fig.haarEmmaWatson}
    \end{figure}
    %pierdolenie na koniec rozdzialu
    Biblioteka OpenCV posiada wytrenowane klasyfikatory, które zostały użyte w pracy magisterkiej. Użyto
    klasyfikatorów do wykrycia twarzy, ust oraz oczu.
    \section{Wyznaczanie stref}\label{sec:wyznaczanieStref}
    sdfsd
    \section{Wykrywanie zmarszczek - detektor Canny}\label{sec:wykrywanieZmarszczek}
    sdfsd
    \section{Wyliczanie wrinkle feature}\label{sec:wyliczanieWrinkleFeature}
    sdfsd
    \section{Algorytm trenowania}\label{sec:algorytmTrenowania}
    sfsd
    \section{Grupowanie danych - FCM}\label{sec:grupowanieDanych}
    sfsd
    \subsection{Wstęp do grupowania danych}
    \subsection{Metoda FCM}
    sdfsd
    \section{Wyznaczanie wieku}\label{sec:wyznaczanieWieku}

    %ZMIESCISC SIE W 25-30 STRONACH

    \chapter{Modyfikacje metody bazowej}

    \section{Odjęcie wybranej strefy}

    \subsection{Zmiana algorytmu względem metody bazowej}

    \section{Zastosowanie metody HOG}
    \subsection{Opis algorytmu HOG}
    \subsection{Zastosowanie w projekcie}

    \section{Metoda HOG oraz grupowanie KNN}
    \subsection{Grupowanie KNN}
    \subsection{Zastosowanie w projekcie}
    %ZMIESCIC ROZDZIAL W 15-20
    \chapter{Badania}

    %opisywac nie tylko wyniki ale tez posrednio co tam lecialo, statystyki, szybkosc dzialania (sredni czas
    %    przetwarzania), zajetosc pamieci
    \chapter{Podsumowanie}



    %%%%%%%%%%%%%%%%%%%%%%%%%%%%%%%%%%%%%%%%%%
    \backmatter
    \pagenumbering{Roman}
    \stepcounter{stronyPozaNumeracja}
    \setcounter{page}{\value{stronyPozaNumeracja}}

    \pagestyle{tylkoNumeryStron}

    %%%%%%%%%%% bibliografia %%%%%%%%%%%%
    \bibliographystyle{plplain}
    \bibliography{bibliografia}

    %%%%%%%%%  DODATKI %%%%%%%%%%%%%%%%%%%

    \begin{appendices}


        \chapter*{Dokumentacja techniczna}

        \chapter*{Spis skrótów i symboli}

        \begin{itemize}
            \item[DNA] kwas deoksyrybonukleinowy (ang. \ang{deoxyribonucleic acid})
            \item[MVC] model -- widok -- kontroler (ang. \ang{model--view--controller})
            \item[$N$] liczebność zbioru danych
            \item[$\mu$] stopnień przyleżności do zbioru
            \item[$\mathbb{E}$] zbiór krawędzi grafu
            \item[$\mathcal{L}$] transformata Laplace'a
        \end{itemize}




        \chapter*{Zawartość dołączonej płyty}

        Do pracy dołączona jest płyta CD z~następującą zawartością:
        \begin{itemize}
            \item praca w~formacie \texttt{pdf},
            \item źródła programu,
            \item zbiory danych użyte w~eksperymentach.
        \end{itemize}

        \listoffigures
        \listoftables

    \end{appendices}


\end{document}


%% Finis coronat opus.