%%%%%%%%%%%%%%%%%%%%%%%%%%%%%%%%%%%%%%%%%%
%                                        %
% Szablon pracy dyplomowej magisterskiej % 
%                                        %
%%%%%%%%%%%%%%%%%%%%%%%%%%%%%%%%%%%%%%%%%%



\documentclass[a4paper,twoside,12pt]{book}
\usepackage[utf8]{inputenc}                                      
\usepackage[T1]{fontenc}  
\usepackage{amsmath,amsfonts,amssymb,amsthm}
\usepackage[british,polish]{babel} 
\usepackage{indentfirst}
\usepackage{lmodern}
\usepackage{graphicx} 
\usepackage{hyperref}
\usepackage{booktabs}
\usepackage{tikz}
\usepackage{pgfplots}
\usepackage{mathtools}
\usepackage[page]{appendix} % toc,
\renewcommand{\appendixtocname}{Dodatki}
\renewcommand{\appendixpagename}{Dodatki}
\renewcommand{\appendixname}{Dodatek}

\usepackage{setspace}
\onehalfspacing


\frenchspacing

\usepackage{listings}
\lstset{
	language={},
	basicstyle=\ttfamily,
	keywordstyle=\lst@ifdisplaystyle\color{blue}\fi,
	commentstyle=\color{gray}
}

%%%%%%%%%

%%%% TODO LIST GENERATOR %%%%%%%%%

%\usepackage{tikz}
%\usepackage{manfnt}   % dangerous sign 
\usepackage{color}
\definecolor{brickred}      {cmyk}{0   , 0.89, 0.94, 0.28}

\makeatletter \newcommand \kslistofremarks{\section*{Uwagi} \@starttoc{rks}}
  \newcommand\l@uwagas[2]
    {\par\noindent \textbf{#2:} %\parbox{10cm}
{#1}\par} \makeatother


\newcommand{\ksremark}[1]{%
{%\marginpar{\textdbend}
{\color{brickred}{[#1]}}}%
\addcontentsline{rks}{uwagas}{\protect{#1}}%
}

\newcommand{\comma}{\ksremark{przecinek}}
\newcommand{\nocomma}{\ksremark{bez przecinka}}
\newcommand{\styl}{\ksremark{styl}}
\newcommand{\ortografia}{\ksremark{ortografia}}
\newcommand{\fleksja}{\ksremark{fleksja}}
\newcommand{\pauza}{\ksremark{pauza `--', nie dywiz `-'}}
\newcommand{\kolokwializm}{\ksremark{kolokwializm}}

%%%%%%%%%%%%%% END OF TODO LIST GENERATOR %%%%%%%%%%%

%%%%%%%%%%%% ZYWA PAGINA %%%%%%%%%%%%%%%
% brak kapitalizacji zywej paginy
\usepackage{fancyhdr}
\pagestyle{fancy}
\fancyhf{}
\fancyhead[LO]{\nouppercase{\it\rightmark}}
\fancyhead[RE]{\nouppercase{\it\leftmark}}
\fancyhead[LE,RO]{\it\thepage}


\fancypagestyle{tylkoNumeryStron}{%
   \fancyhf{} 
   \fancyhead[LE,RO]{\it\thepage}
}

\fancypagestyle{NumeryStronNazwyRozdzialow}{%
   \fancyhf{} 
   \fancyhead[LO]{\nouppercase{\it\rightmark}}
   \fancyhead[RE]{\nouppercase{\it\leftmark}}
   \fancyhead[LE,RO]{\it\thepage}
}


%%%%%%%%%%%%% OBCE WTRETY  
\newcommand{\obcy}[1]{\emph{#1}}
\newcommand{\ang}[1]{{\selectlanguage{british}\obcy{#1}}}
%%%%%%%%%%%%%%%%%%%%%%%%%%%%%

% polskie oznaczenia funkcji matematycznych
\renewcommand{\tan}{\operatorname {tg}}
\renewcommand{\log}{\operatorname {lg}}

% jeszcze jakies drobiazgi

\newcounter{stronyPozaNumeracja}

\newcommand{\hcancel}[1]{%
    \tikz[baseline=(tocancel.base)]{
        \node[inner sep=0pt,outer sep=0pt] (tocancel) {#1};
        \draw[red] (tocancel.south west) -- (tocancel.north east);
    }%
}%

\newcommand{\miesiac}{%
  \ifcase\the\month
  \or styczeń% 1
  \or luty% 2
  \or marzec% 3
  \or kwiecień% 4
  \or maj% 5
  \or czerwiec% 6
  \or lipiec% 7
  \or sierpień% 8
  \or wrzesień% 9
  \or październik% 10
  \or listopad% 11
  \or grudzień% 12
  \fi}


%%%%%%%%%%%%%%%%%%%%%%%%%%%%%%%%%%%%%%%%%%%%%%
%%%%%%%%%%%%%%%%%%%%%%%%%%%%%%%%%%%%%%%%%%%%%%
%%%%%%%%%%%%%%%%%%%%%%%%%%%%%%%%%%%%%%%%%%%%%%
%%%%%%%%%%%%%%%%%%%%%%%%%%%%%%%%%%%%%%%%%%%%%%
%%%%%%%%%%%%%%%%%%%%%%%%%%%%%%%%%%%%%%%%%%%%%%
%%%%%%%%%%%%%%%%%%%%%%%%%%%%%%%%%%%%%%%%%%%%%%
%%%%%%%%%%%%%%%%%%%%%%%%%%%%%%%%%%%%%%%%%%%%%%
%%%%%%%%%%%%%%%%%%%%%%%%%%%%%%%%%%%%%%%%%%%%%%


\newcommand{\autor}{Mateusz Trzeciak}
\newcommand{\promotor}{dr inż. Karolina Nurzyńska}
\newcommand{\tytul}{Określenie wieku twarzy na podstawie tekstury}


\begin{document}
%\kslistofremarks 
	
%%%%%%%%%%%%%%%%%%  STRONA TYTULOWA %%%%%%%%%%%%%%%%%%%
\pagestyle{empty}
\sffamily

\noindent

\begin{center}
    \large
    Politechnika Śląska\\
    Wydział Automatyki, Elektroniki i~Informatyki \\ 
    kierunek: informatyka
\end{center}

\vfill\vfill
\begin{center}
    \large
    \autor
\end{center}

\vfill
\begin{center}
   \LARGE\bfseries \tytul
\end{center}

\vfill
\begin{center}
    \large
    praca dyplomowa magisterska
\end{center}

\vfill\vfill\vfill
\begin{center}
    \large
    \begin{tabular}{ll}
    promotor: & \promotor \\
   % konsultant: & \konsultant \\ % jezeli nie ma, zakomentowac
    \end{tabular}

\end{center}

\vfill
\begin{center}
\large
   Gliwice,  \miesiac\ \the\year
\end{center}

\cleardoublepage
 

\rmfamily
\normalfont

%%%%%%%%%%%%%%%%%%%%% oswiadczenie o udostępnianiu pracy dyplomowej %%%%%%%%%%%%%%%%%%%
\cleardoublepage

\begin{flushright}
załącznik nr 2 do zarz. nr 97/08/09 
\end{flushright}

\vfill  

\begin{center}
\Large\bfseries Oświadczenie
\end{center}

\vfill

Wyrażam  zgodę / Nie wyrażam zgody*  na  udostępnienie  mojej  pracy  dyplomowej / rozprawy doktorskiej*.

\vfill

Gliwice, dnia \today

\vfill

\rule{0.5\textwidth}{0cm}\dotfill 

\rule{0.5\textwidth}{0cm}
\begin{minipage}{0.45\textwidth}
{\begin{center}(podpis)\end{center}}
\end{minipage} 

\vfill

\rule{0.5\textwidth}{0cm}\dotfill 

\rule{0.5\textwidth}{0cm}
\begin{minipage}{0.45\textwidth}
{\begin{center}\rule{0mm}{5mm}(poświadczenie wiarygodności podpisu przez Dziekanat)\end{center}}
\end{minipage}


\vfill

* podkreślić właściwe

 


%%%%%%%%%%%%%%%%%%%%% oswiadczenie promotora o spełnieniu wymagań formalnych %%%%%%%%%%%%%%%%%%%
\cleardoublepage

\rule{1cm}{0cm}

\vfill  

\begin{center}
\Large\bfseries Oświadczenie promotora
\end{center}

\vfill

Oświadczam, że praca „\tytul” spełnia wymagania formalne pracy dyplomowej magisterskiej.

\vfill



\vfill

Gliwice, dnia \today

\rule{0.5\textwidth}{0cm}\dotfill 

\rule{0.5\textwidth}{0cm}
\begin{minipage}{0.45\textwidth}
{\begin{center}(podpis promotora)\end{center}}
\end{minipage} 

\vfill

%\rule{0.5\textwidth}{0cm}\dotfill 
%
%\rule{0.5\textwidth}{0cm}
%\begin{minipage}{0.45\textwidth}
%{\begin{center}\rule{0mm}{5mm}(poświadczenie wiarygodności podpisu przez Dziekanat)\end{center}}
%\end{minipage}
%
%
%\vfill

 

\cleardoublepage


%%%%%%%%%%%%%%%%%% SPIS TRESCI %%%%%%%%%%%%%%%%%%%%%%
\pagenumbering{Roman}
\pagestyle{tylkoNumeryStron}
\tableofcontents

%%%%%%%%%%%%%%%%%%%%%%%%%%%%%%%%%%%%%%%%%%%%%%%%%%%%%
\setcounter{stronyPozaNumeracja}{\value{page}}
\mainmatter
\pagestyle{NumeryStronNazwyRozdzialow}

%%%%%%%%%%%%%% wlasciwa tresc pracy %%%%%%%%%%%%%%%%%

\chapter{Wstęp}

\begin{itemize}
\item wprowadzenie w problem/zagadnienie 
\item osadzenie problemu w dziedzinie 
\item cel pracy 
\item zakres pracy 
\item zwięzła charakterystyka rozdziałów 
\item jednoznaczne określenie wkładu autora
\end{itemize}
Z weryfikacją wieku danej osoby można się spotkać przed wejściem do niektórych miejsc np. klub nocny. Większość osób musi okazać ważny dowód osobisty. Co generuje duże kolejki do wejścia. Aplikację analizujące wiek na podstawie obrazu twarzy z kamery, przed wejściem do takich miejsc znacząco usprawniłyby weryfikację więku.
Określenie wieku twarzy na podstawie tekstury jest związane bezpośrednio z rozpoznawaniem obrazu.


\chapter{[Analiza tematu]}

\begin{itemize}
\item analiza tematu
\item wprowadzenie do dziedziny (\ang{state of the art}) – sformułowanie problemu, 
\item poszerzone studia literaturowe, przegląd literatury tematu (należy wskazać źródła wszystkich informacji zawartych w pracy)
\item opis znanych rozwiązań, algorytmów, osadzenie pracy w kontekście
\item Tytuł rozdziału jest często zbliżony do tematu pracy. 
\item Rozdział jest wysycony cytowaniami do literatury \cite{bib:artykul,bib:ksiazka,bib:konferencja}.
\end{itemize}




\chapter{[Przedmiot pracy]}

\begin{itemize}
\item rozwiązanie zaproponowane przez dyplomanta
\item analiza teoretyczna rozwiązania
\item uzasadnienie wyboru zastosowanych metod, algorytmów, narzędzi
\end{itemize}

\chapter{Badania}

 

Rozdział przedstawia przeprowadzone badania. Jest to zasadnicza część i~musi wyraźnie dominować w~pracy.
Badania i analizę wyników należy przeprowadzić, tak jak jest przyjęte w środowisku naukowym (na przykład korzystanie z danych benchmarkowych, walidacja krzyżowa, zapewnienie powtarzalności testów itd). 

\section{Metodyka badań}

\begin{itemize}
\item opis metodyki badań
\item opis stanowiska badawczego (opis interfejsu aplikacji badawczych -- w~załączniku)
\end{itemize}


\section{Zbiory danych}

\begin{itemize}
\item opis danych
\end{itemize}


\section{Wyniki}

\begin{itemize}
\item prezentacja wyników, opracowanie i poszerzona dyskusja  wyników, wnioski
\end{itemize}


\begin{figure}
\centering
\begin{tikzpicture}
\begin{axis}[
    y tick label style={
        /pgf/number format/.cd,
            fixed,   % po zakomentowaniu os rzednych jest indeksowana wykladniczo
            fixed zerofill, % 1.0 zamiast 1
            precision=1,
        /tikz/.cd
    },
    x tick label style={
        /pgf/number format/.cd,
            fixed,
            fixed zerofill,
            precision=2,
        /tikz/.cd
    }
]
\addplot [domain=0.0:0.1] {rnd};
\end{axis} 
\end{tikzpicture}
\caption{Podpis rysunku po rysunkiem.}
\label{fig:2}
\end{figure}

\begin{table}
\centering
\caption{Opis tabeli nad nią.}
\label{id:tab:wyniki}
\begin{tabular}{rrrrrrrr}
\toprule
	         &                                     \multicolumn{7}{c}{metoda}                                      \\
	         \cmidrule{2-8}
	         &         &         &        \multicolumn{3}{c}{alg. 3}        & \multicolumn{2}{c}{alg. 4, $\gamma = 2$} \\
	         \cmidrule(r){4-6}\cmidrule(r){7-8}
	$\zeta$ &     alg. 1 &   alg. 2 & $\alpha= 1.5$ & $\alpha= 2$ & $\alpha= 3$ &   $\beta = 0.1$  &   $\beta = -0.1$ \\
\midrule
	       0 &  8.3250 & 1.45305 &       7.5791 &    14.8517 &    20.0028 & 1.16396 &                       1.1365 \\
	       5 &  0.6111 & 2.27126 &       6.9952 &    13.8560 &    18.6064 & 1.18659 &                       1.1630 \\
	      10 & 11.6126 & 2.69218 &       6.2520 &    12.5202 &    16.8278 & 1.23180 &                       1.2045 \\
	      15 &  0.5665 & 2.95046 &       5.7753 &    11.4588 &    15.4837 & 1.25131 &                       1.2614 \\
	      20 & 15.8728 & 3.07225 &       5.3071 &    10.3935 &    13.8738 & 1.25307 &                       1.2217 \\
	      25 &  0.9791 & 3.19034 &       5.4575 &     9.9533 &    13.0721 & 1.27104 &                       1.2640 \\
	      30 &  2.0228 & 3.27474 &       5.7461 &     9.7164 &    12.2637 & 1.33404 &                       1.3209 \\
	      35 & 13.4210 & 3.36086 &       6.6735 &    10.0442 &    12.0270 & 1.35385 &                       1.3059 \\
	      40 & 13.2226 & 3.36420 &       7.7248 &    10.4495 &    12.0379 & 1.34919 &                       1.2768 \\
	      45 & 12.8445 & 3.47436 &       8.5539 &    10.8552 &    12.2773 & 1.42303 &                       1.4362 \\
	      50 & 12.9245 & 3.58228 &       9.2702 &    11.2183 &    12.3990 & 1.40922 &                       1.3724 \\
\bottomrule
\end{tabular}
\end{table}  



\chapter{Podsumowanie}
\begin{itemize}
\item syntetyczny opis wykonanych prac
\item wnioski
\item możliwość rozwoju, kontynuacji prac, potencjalne nowe kierunki
\item Czy cel pracy zrealizowany? 
\end{itemize}


%%%%%%%%%%%%%%%%%%%%%%%%%%%%%%%%%%%%%%%%%%
\backmatter
\pagenumbering{Roman}
\stepcounter{stronyPozaNumeracja}
\setcounter{page}{\value{stronyPozaNumeracja}}

\pagestyle{tylkoNumeryStron}

%%%%%%%%%%% bibliografia %%%%%%%%%%%%
\bibliographystyle{plplain}
\bibliography{bibliografia}

%%%%%%%%%  DODATKI %%%%%%%%%%%%%%%%%%% 

\begin{appendices} 


\chapter*{Dokumentacja techniczna}

\chapter*{Spis skrótów i symboli}

\begin{itemize}
\item[DNA] kwas deoksyrybonukleinowy (ang. \ang{deoxyribonucleic acid})
\item[MVC] model -- widok -- kontroler (ang. \ang{model--view--controller}) 
\item[$N$] liczebność zbioru danych
\item[$\mu$] stopnień przyleżności do zbioru
\item[$\mathbb{E}$] zbiór krawędzi grafu
\item[$\mathcal{L}$] transformata Laplace'a 
\end{itemize}


 

\chapter*{Zawartość dołączonej płyty}

Do pracy dołączona jest płyta CD z~następującą zawartością:
\begin{itemize}
\item praca w~formacie \texttt{pdf},
\item źródła programu,
\item zbiory danych użyte w~eksperymentach.
\end{itemize}

\listoffigures
\listoftables
	
\end{appendices}


\end{document}


%% Finis coronat opus.